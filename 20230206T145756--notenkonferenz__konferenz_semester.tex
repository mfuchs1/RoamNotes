% Created 2023-03-07 Di 07:14
% Intended LaTeX compiler: pdflatex
\documentclass[pdftex,a4paper,12pt,bibliography=totoc]{scrartcl}
                              \usepackage[top=2.5cm, bottom=2.5cm, left=4.0cm, right=3.0cm]{geometry}
\usepackage{nicefrac}
\usepackage{amsmath}
%\usepackage{ucs}
\usepackage[germanb]{babel}
\usepackage[utf8]{inputenc}
\usepackage[T1]{fontenc}
\usepackage{textcomp}
\usepackage[colorlinks=true,urlcolor=blue,linkcolor=green]{hyperref}
\RequirePackage[ngerman=ngerman-x-latest]{hyphsubst}

\usepackage[style=authoryear-ibid,backend=biber,dashed=false,isbn=false]{biblatex}
\DeclareLanguageMapping{ngerman}{ngerman-apa}

\makeatletter
\renewbibmacro*{bbx:editor}[1]{%
\ifthenelse{\ifuseeditor\AND\NOT\ifnameundef{editor}}
{\ifthenelse{\iffieldequals{fullhash}{\bbx@lasthash}\AND
\NOT\iffirstonpage\AND
\(\NOT\boolean{bbx@inset}\OR
\iffieldequalstr{entrysetcount}{1}\)}
{\bibnamedash}
{\printnames{editor}%
\setunit{\addspace}% GEÄNDERT
\usebibmacro{bbx:savehash}}%
\printtext[parens]{\usebibmacro{#1}}% GEÄNDERT
\clearname{editor}%
\setunit{\addspace}}%
{\global\undef\bbx@lasthash
\usebibmacro{labeltitle}%
\setunit*{\addspace}}%
\usebibmacro{date+extradate}}
\makeatother

\DeclareBibliographyDriver{incollection}{%
\usebibmacro{bibindex}%
\usebibmacro{begentry}%
\usebibmacro{author/translator+others}%
\setunit{\labelnamepunct}\newblock
\usebibmacro{title}%
\newunit
\printlist{language}%
\newunit\newblock
\usebibmacro{byauthor}%
\newunit\newblock
\usebibmacro{in:}%
\begingroup% NEU
\renewbibmacro*{date+extradate}{}% NEU
\usebibmacro{editor+others}% NEU
\newunit{\addcolon\addspace}\newblock% NEU
\endgroup% NEU
\usebibmacro{maintitle+booktitle}%
\newunit\newblock
%  \usebibmacro{byeditor+others}%
%  \newunit\newblock
\newunit\newblock
\usebibmacro{chapter+pages}%
\printfield{edition}%
\newunit
\iffieldundef{maintitle}
{\printfield{volume}%
\printfield{part}}
{}%
\newunit
\printfield{volumes}%
\newunit\newblock
\usebibmacro{series+number}%
\newunit\newblock
\printfield{note}%
\newunit\newblock
\usebibmacro{publisher+location+date}%
%\newunit\newblock
%\usebibmacro{chapter+pages}%
\newunit\newblock
\iftoggle{bbx:isbn}
{\printfield{isbn}}
{}%
\newunit\newblock
\usebibmacro{doi+eprint+url}%
\newunit\newblock
\usebibmacro{addendum+pubstate}%
\newunit\newblock
\usebibmacro{pageref}%
\usebibmacro{finentry}}

% \DeclareBibliographyDriver{article}{
%   \usebibmacro{bibindex}%
%   \usebibmacro{begentry}%
%   \usebibmacro{author/translator+others}%
%   \setunit{\labelnamepunct}\newblock
%   \usebibmacro{title}%
%   \newunit
%   \printlist{language}%
%   \newunit\newblock
%   \usebibmacro{byauthor}%
%   \newunit\newblock
%   \usebibmacro{byeditor+others}%
%   \newunit\newblock
%   \printfield{version}%
%   \newunit\newblock
% %  \usebibmacro{in:}%   SO KLAPPT DAS!
%   \usebibmacro{journal+issuetitle}%
%   \newunit\newblock
%   \printfield{note}%
%   \setunit{\bibpagespunct}%
%   \printfield{pages}
%   \newunit\newblock
%   \printfield{issn}%
%   \newunit\newblock
%   \printfield{doi}%
%   \newunit\newblock
%   \usebibmacro{eprint}
%   \newunit\newblock
%   \usebibmacro{url+urldate}%
%   \newunit\newblock
%   \printfield{addendum}%
%   \newunit\newblock
%   \usebibmacro{pageref}%
%   \usebibmacro{finentry}
% }

\DefineBibliographyStrings{german}{%
byeditor = {Hrsg\adddot},%
byeditor = {Hrsg\adddot},%
andothers={et \addabbrvspace al\adddot},
}

%\DeclareNameAlias{sortname}{last-first}
%\DeclareNameAlias{default}{last-first}
\DeclareNameAlias[incollection]{editor}{default}
%\DeclareFieldFormat{namelast}{\mkbibacro{#1}}
\DeclareFieldFormat[incollection]{title}{\mkbibitalic{#1}}
\DeclareFieldFormat[incollection]{booktitle}{#1}
\DeclareFieldFormat[incollection]{pages}{\parentext{#1}}
\DeclareFieldFormat[incollection]{editor}{#1\addcolon}
\DeclareFieldFormat[article]{title}{\mkbibitalic{#1}}
\DeclareFieldFormat[article]{journaltitle}{#1}
\DeclareFieldFormat[article]{urldate}{\brackettext{#1}}
\DeclareFieldFormat[article]{note}{#1\addcolon\addspace}
\DeclareFieldFormat[collection]{urldate}{\brackettext{#1}}
\DeclareFieldFormat[collection]{note}{#1\addcolon\addspace}
\DeclareFieldFormat[misc]{urldate}{\brackettext{#1}}
\DeclareFieldFormat[misc]{note}{#1\addcolon\addspace}


\AtBeginBibliography{%
\renewcommand{\nametitledelim}{\addcolon\space}
\renewcommand{\finalnamedelim}{\addcomma\space}

}
%\renewcommand{\mkbibnamefamily}[1]{\textsc{#1}}
\renewcommand{\labelnamepunct}{\addcolon\addspace}
\renewcommand{\nameyeardelim}{\addcomma\space}
\renewcommand{\subtitlepunct}{\adddot\space}




%\renewcommand{\nametitledelim}{\addcolon\addspace}
%\addbibresource{Literatur.bib} %% Einbinden der bib-Datei
\bibliography{../Bibliography/Literatur}
\providecommand{\apashortdash}{-}


%\makeatletter
%\renewcommand\@biblabel[1]{}
%\makeatother
\usepackage[babel,german=quotes]{csquotes}
\usepackage{url}
\urlstyle{rm}
\usepackage[pdftex]{graphicx}
\usepackage{hyperref}
\usepackage{cjhebrew}
\renewcommand{\figurename}{Abbildung}
\usepackage{pdfpages}
\renewcommand{\familydefault}{\rmdefault}
\usepackage{times}
\addtokomafont{sectioning}{\rmfamily}
\usepackage{setspace}
%\renewcommand{\familydefault}{\sfdefault}
%\usepackage{helvet}
%\usepackage{lmodern}
\usepackage{booktabs}
\usepackage{ragged2e}
\RequirePackage{processkv}
\usepackage{parcolumns}
\usepackage{blindtext}
\setcounter{tocdepth}{3}
\setcounter{secnumdepth}{3}
\usepackage{xpatch}



%Kopfzeile
\usepackage[draft=false,automark,headsepline,plainheadsepline]{scrlayer-scrpage}
%\KOMAoptions{onpsinit=\linespread{1}\selectfont
\pagestyle{scrheadings}
\clearmainofpairofpagestyles
\clearplainofpairofpagestyles
\ihead{\headmark}
\ohead{\pagemark}
\automark{section}
\onehalfspacing


\title{Notenkonferenz}
\begin{document}


\begin{center}
 \LARGE
 \textbf{Notenkonferenz}
\end{center}

\normalsize
\section{Begrüßung}
\label{sec:org472ccbb}

Gratulation an alle Geburtstagskinder
\newline
Danke an alle für die Supplierungen; Deisl A. kann es noch nicht sagen, wann sie kommt. Birgit selber nicht ganz fit. Starke Antibiotika - Magen.
\newline
Wer für Astrid suppliert - ist bezahlt; bis Astrid wieder zurück ist.
\newline
Corona hat Spuren hinterlassen: Empfindlichkeiten der Eltern, etc. Beschwerden sollten besser bei Birgit landen, als direkt bei Fr. Kaserer. Versuchen, Kinder nicht bloßzustellen, sondern sie wachsen lassen.

\section{Corona und seine Auswirkungen}
\label{sec:orgaca8a93}
Wie war das Klima in der Klasse? Daran denkne Schüler. Nicht wenige Eltern sind nicht gleich einsichtig, wenn Kinder abgestuft werden. Erst Abstufung, wenn Förderungen angeboten werden. Gute Heftführung \ldots{}
\newline
Z.B.: SA Mathematik in der 3. Klasse; Hannes: SA nicht gut (muss wiederholt werden). Leichte Schulstufe - mit Schnellhefter. Vorher 3 Hefte mit blauem Umschlag. Mit Schnellhefter probiert, bei manchen hats gut geklappt, bei anderen nicht. Eigenverantwortung. Eltern beschweren sich, va von den leistungsschwachen Schülern.
\newline
Julia: SÜ Heft + Ordner für HÜ; \ldots{}

\section{Vetragverlängerung}
\label{sec:org02ab62b}
Herabsetzungen bis 13.3.2023 

\section{Klassenkonferenzen}
\label{sec:org40d0a5d}
\begin{itemize}
\item 1a kein NG; kein WZ
\item 1b kein NG; ein WZ für Lucian (seine Mutter hat zwei taube Ohren)
\item 2a kein NG; Änderungen noch nicht; Betragen okay
\item 2b Vivien NG in allen drei Hauptfächern + BU; Betragen okay; Abstufungen (?); Vivine WZ wegen Schulschwänzen (nicht im Unterricht, aber im Cafehaus, unentschuldigt).
\item 2c ein NG Ivana in D; auch Leistungsabfall in E; einige Z, kein WZ.
\item 3a NG Adrian in GW; kein WZ;
\item 3b Mathematik: Schönauer und Höllbacher Änderung; sechs WZ (s.u.); \ldots{}
\item 3c kein NG, Umstufung nein; zwei Z für Slatko und Maid (?), eher WZ? eigentlich WZ wegen unentschuldigtes Fernbleiben vom Unterricht; Belügen des Lehrers; \ldots{}
\item 4a kein NG; Leistungsänderung Johanna Strubreiter in E;
\item 4b Carlos ist von Schule abgemeldet; Umstufungen keine; Markus Kaindl nach AHS?; kein NG; Markus aus disziplinären Gründen nicht nach Wien; \ldots{}
\end{itemize}


Wolfi: wegen angeblicher mangelnder Unterstützung als "Türkenhasser" bechimpft; Mutter direkt in Direktion; \ldots{}
\newline
Erwin: Ein WZ ist kein Problem; ein NZ wird haarig. 
\newline
Julia: türkische Mädchen "So dürfen sie nicht mit uns reden." - Mädchen erhalten ein WZ.
\newline
Christoph: Wenn das schwarze Buch leer ist, sind Diskussionen umsonst. 
\newline
Irmi: Was ist schulschwänzen? unentschuldigtes Fernbleiben vom Unterricht; \ldots{} Wenn man Schüler beim Spar sehe, der offiziell krank und entschuldigt ist? Vivien kommt in die Schule, erscheint nicht im Unterricht, denn retour in der Garderobe.
\newline
Birgit: Nachfrage bei Juristen; Vivien nicht nur schwänzen, sondern generelles Verhalten gegenüber der Schule.
\newline
Erwin: nur wenn es während der Unterrichtszeit ist. Alles andere wird haarig.
\newline
Hannes: alles, was jemand zur Genesung dient kann er tun.

\section{Notenkonferenz}
\label{sec:org7147ae9}

\subsection{Betragensnoten 3B}
\label{sec:org4bb2e20}

\begin{itemize}
 \item Maximilian Fischhofer
 \item Lukas Wagenhifer
 \item Leon Keser
 \item Thomas Krallinger
 \item Florian Steiner
\end{itemize}

WZ, weil: seit Beginn des Semesters massivste Störungen im Unterricht. Nach mehrmaligen Ermahnen keine Besserungen; zum Elternsprechtag Eltern herinnen, versprachen Verbesserung. IN der SchoolFox Nachricht wurde WZ angedroht. Keine Verbesserung, also WZ.

\section{Allfälliges}
\label{sec:org7efa80f}
Nach den Ferien sind Schuleinschreibungen: 3x 1. Klassen; Eva die 3. KV
\newline
Schulnachrichten ausdrucken.
\newline
Freitag 5. Stunde, Unterrichtsende.
\newline
Programm steht für Faschingsdienstag - eintragen.
\newline
Eva: neue Homepage, gemeinsam mit VS Golling; Über Bildungsdirektion; VS hat ein Angebot. Neue Homepage kostet mehr als 3000 Euro. BM und Gernot wissen Bescheid. 

\section{Ergänzungen (durch Manuela Singer)}
\label{sec:org94a60f2}

Förderschiene: 1. Klasse - 1. Std / Woche für alle im Stundenplan eintragen als fixe Förderstunde. z.B. 4 Std. im Block D oder M.
\newline
Verhalten der Eltern (zunehmend ein Problem); deshlab wichtig: ein guter Zusammenhalt im Kollegium
\newline
Porfessionelle Distanz bezüglich der Eltern wäre günstig. Warum mischen sich Eltern so viel ein?
\newline
SA-Termine für das 2. Semester festlegen.
\end{document}
